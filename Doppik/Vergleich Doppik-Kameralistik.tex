% !TeX encoding = ISO-8859-1
\documentclass[
10pt,			%Schriftgr��e
paper=a4,		%Papierformat
ngerman,		%Deutsche Silbentrennung
BCOR=0pt,		%Bindekorrektureinstellung=AN (15mm) Bindekorrektureinstellung=AUS > 0pt
%twoside,		%Bindekorrektur f�r zweiseitigen Druck
DIV=calc,		%Satzspiegelberechnung
headinclude,	%Kopfzeile bei Satzspiegelberechnung mit einbeziehen
headsepline,	%Trennlinie unter der Kopfzeile
numbers=noenddot,
]{scrreprt}
\usepackage[T1]{fontenc}
\usepackage{lmodern}
\usepackage[sc]{mathpazo}

% Satzspiegel wird neu berechnet
\typearea[current]{calc}

\usepackage[ansinew]{inputenc}

% Paket f�r besseres Schriftbild
\usepackage{microtype}


% Allgemeine Pakete:

\usepackage{xcolor}
\usepackage[ngerman]{babel}
\usepackage{relsize}
\usepackage{paralist}
\usepackage{mdwlist}
\usepackage{typearea}
\usepackage{setspace}
\usepackage{textcomp}
\usepackage{marvosym}

\usepackage{fancybox}
\usepackage{array}
\usepackage{amsfonts,amsmath,amssymb}
\usepackage{pifont}
\usepackage{calc}
\usepackage{ifthen}
\usepackage{tikz}
\usepackage{pgf-pie}
\usepackage{pgfplots}
\pgfplotsset{compat=1.8}


\usetikzlibrary{shapes,arrows,shadows}

\usepackage{scrpage2}

% Pakete f�r Abbildungen
\usepackage{graphicx}
\usepackage{floatrow}
\usepackage{floatpag}



% Fu�notengefrickel muss vor HYPERREF geladen werden
\usepackage[bottom,flushmargin,hang,multiple]{footmisc}

%Hyperxmp-Paket einbinden f�r PDF-Metadaten
\usepackage{hyperxmp} 

% Rahmen um Hyperlinks unterbinden
\usepackage[bookmarks=true,pdftoolbar=true,pdfmenubar=true,pdfstartview={FitH},citecolor=magenta,hidelinks,breaklinks=true]{hyperref}

\hypersetup{ %
pdftitle={Gasmarktliberalisierung im Baltikum}
pdfauthor={Pascal Bernhard}
pdfproducer={Texlive-LaTeX 2013 on Sabayon GNU/Linux 3.12 & 3.13}
pdfcopyright={Creative Commons 3.0 CC-BY}
pdflicenseurl={https://creativecommons.org/licenses/by/3.0/de/legalcode}
pdfmetalang={de-DE}
}




%%% Auf tats�chlichen Namen des Abschnitts verlinken
\usepackage{titleref}
\usepackage{nameref}



% Abst�nde zwischen �berschrift und Textk�rper modifizieren [PAKET NICHT INSTALLIERT!!!!]
%\titlesec

% Gleiche Schriftart f�r Hyperlinks
\urlstyle{same}


%  Gefrickel um URL-Links vern�nftig umzubrechen
\makeatletter
\g@addto@macro\UrlBreaks{
  \do\a\do\b\do\c\do\d\do\e\do\f\do\g\do\h\do\i\do\j
  \do\k\do\l\do\m\do\n\do\o\do\p\do\q\do\r\do\s\do\t
  \do\u\do\v\do\w\do\x\do\y\do\z\do\&\do\1\do\2\do\3
  \do\4\do\5\do\6\do\7\do\8\do\9\do\0}
% \def\do@url@hyp{\do\-}

% Hiermit soll einer �bervolle Box verhindert werden -- funktioniert sogar irgendwie
\g@addto@macro\UrlSpecials{\do\/{\mbox{\UrlFont/}\hskip 0pt plus 1pt}}
\makeatother

% Hier kommen Optionen der KOMA-Klasse zu �berschriften
\KOMAoptions{headings=small}

% Einstellungen fpr Kopfzeilen
\ihead{\leftmark} 
\ohead{\rightmark} 
\chead{} 
\pagestyle{scrheadings} 
\automark[section]{chapter}

\setfootnoterule[0.8pt]{10cm} % H�he und L�nge des Trennstriches f�r Fu�noten anpassen
\setkomafont{caption}{\sffamily} % Bezeichnungen von Bilder werden serifenlos und fett gesetzt
\setkomafont{captionlabel}{\sffamily\bfseries} % Beschreibungen von Bilder werden serifenlos und fett gesetzt

\renewcommand{\headfont}{\small\sffamily\mdseries} % Schriftart der Kopfzeile �ndern



%%%----------------------------------------------------------------------
\definecolor{dunkelgrau}{gray}{0.20}
\definecolor{hellgrau}{gray}{0.40}
\definecolor{MidnightBlue}{RGB}{0,103,149}
\definecolor{NavyBlue}{RGB}{0,110,184}
\definecolor{MidBlue}{rgb}{0.173,0.212,0.597}
\definecolor{Dschungel}{cmyk}{0.99,0,0.52,0.2}


%%% Palatino ben�rtigt gr��eren Raum zwischen den Zeilen
\linespread{1.05}


\onehalfspacing
\typearea[current]{calc}



%%% Schriften umdefinieren:

\newcommand{\changefont}[3]{
\fontfamily{#1} \fontseries{#2} \fontshape{#3} \selectfont}

\newcommand{\textemph}{\textsl{\textbf}}
%%%----------------------------------------------------------------------
\begin{document}

\begin{spacing}{1}

%%% Deckblatt Start
\thispagestyle{empty}


\begin{center}
\Large{Hochschule f�r Wirtschaft und Recht Berlin}\\
\end{center}
 
 
\begin{center}
\Large{Bachelor-Studiengang }
\end{center}
\begin{verbatim}
 
 
\end{verbatim}
\begin{center}
\textbf{\LARGE{\changefont{ppl}{b}{n}
\sffamily{Hausarbeit}}}
\end{center}
\begin{verbatim}
 
 
\end{verbatim}
\begin{center}
\textbf{\changefont{ppl}{b}{n}
\sffamily{im Studiengang Politik}}
\end{center}
\begin{verbatim}
 
 
\end{verbatim}
 
\begin{flushleft}
\begin{tabular}{lll}

\textbf{Thema:} & & Doppik \& Kameralistik --\\
& & \\
& & XXX\\
& & \\
& & \\
& & \\



\textbf{eingereicht von:} & & Aida Kaloki \flq{}aidakal@yahoo.com\frq{}\\
& & \\
& & \\
\textbf{eingereicht am:} & & 16. Juni 2016\\
& & \\
& & \\
\textbf{Betreuer:} & & Herr Prof. Dr. XX\\
\textbf{Betreuer:} & & Frau Prof. Dr. XX
\end{tabular}

\newpage

\thispagestyle{empty}


%\textbf{Diese Diplomarbeit widme ich Frau Andrea Volmary und Herrn Kai-Uwe Christoph, ohne deren Hilfe mein Studienabschluss nicht m�glich gewesen w�re.}

\end{flushleft}

\end{spacing}
 
%Deckblatt Ende


%%%Palatino als Standardschrift
\renewcommand{\rmdefault}{ppl}

\renewcommand\thesubsection{\thesection.\alph{subsection}}

%%%Inhaltsverzeichnis----------------------------------------------------------------------

\cleardoublepage
\begingroup
\renewcommand*{\chapterpagestyle}{empty}
\pagestyle{empty}
\tableofcontents
\clearpage
\endgroup

%%%----------------------------------------------------------------------
\setlength{\parindent}{30pt}
\setlength{\parskip}{0pt}


%%%-----------------------------------------------------------------------

%%%Einleitung



\chapter{Einleitung}


\chapter{Thema \& Fragestellung}


\chapter{Haushaltswesen}

Alle mit dem Haushalt einer �ffentliche Gebietsk�rperschaft zusammenh�ngenden Bereich werden unter dem Oberbegriff \textsl{Haushaltswesen} gefasst. Hierzu z�hlen unter anderem die Haushaltsplanung, der Haushaltsvollzug und die Haushaltssteuerung. Je nach Rechnungsstil kann das Haushaltswesen einen sogenannten \textsl{kameralen}, \textsl{erweiterten kameralen}, oder \textsl{doppischen} Charakter haben. Den Kern des Haushaltswesen bildet unabh�ngig vom angewandten Rechnungsstil der Haushaltsplan


\section{Budgetrecht}



\section{Entwicklungen im Haushaltswesen}


\subsection{Neues Steuerungsmodell}


\section{Rechnungswesen}


\chapter{Kameralistische Haushaltsf�hrung}


\chapter{Doppik}


\chapter{Vergleich der Haushaltssysteme}


\section{Vor- und Nachteile der Haushaltssysteme}


\section{Haushaltsgrunds�tze}



\end{document}